\documentclass{article}
\usepackage{fullpage}
\usepackage{enumitem}
\usepackage{mathtools}
\usepackage{amsmath}
\usepackage{titling}

\everymath{\displaystyle}

\begin{document}
\setlength{\droptitle}{-7em}


\title{SICP Exercise 1.13}
\author{Cameron Franz}

\newcommand{\Fib}{\text{Fib}}

\maketitle
\noindent
\textbf{Fibonacci Definition}: \\ \\
$
	\Fib(n) = \begin{cases}
		0, & n=0 \\
		1, & n=1 \\
		\Fib(n-2) + \Fib(n-1), & n>1 
	\end{cases}
$
\\ \\
\textbf{Assertion}: \\ \\
$
	\Fib(n) = \frac{\phi^n-\psi^n}{\sqrt{5}}, \text{ where } \phi = \frac{1 + \sqrt{5}}{2} \text{ and } \psi = \frac{1 - \sqrt{5}}{2}
$
\\ \\
\textbf{Proof}: \\ \\
$
\phi \cdot \psi = \frac{1 - 5}{4} = -1
\\ \\
\phi + \psi = \frac{2}{2} = 1
\\ \\
\Fib(0) = \frac{\phi^0-\psi^0}{\sqrt{5}} = 0
\\ \\
\Fib(1) = \frac{\phi^1-\psi^1}{\sqrt{5}} = \frac{\sqrt{5}}{\sqrt{5}} =1
\\ \\
\begin{aligned}
\Fib(n) & = \frac{\phi^n-\psi^n}{\sqrt{5}} \\ 
		& = \frac{\phi^{n-1}\phi - \psi^{n-1}\psi}{\sqrt{5}} \\
		& = \frac{\phi^{n-1}(1-\psi)-\psi^{n-1}(1-\phi)}{\sqrt{5}} \\
		& = \frac{\phi^{n-1}-\phi^{n-1}\psi - \psi^{n-1}+\psi^{n-1}\phi}{\sqrt{5}} \\
		& = \frac{\phi^{n-1} + \phi^{n-2}-\psi^{n-1}-\psi^{n-2}}{\sqrt{5}} \\
		& = \frac{\phi ^ {n-1} - \psi ^ {n-1}}{\sqrt{5}} + \frac{\phi ^ {n-2} - \psi ^ {n-2}}{\sqrt{5}} \\
		& = \Fib(n-2) + \Fib(n-1)\\
\end{aligned}
$
\\ \\
\textbf{Note} \\ \\
Because $\forall n > 0,  |\psi^n| < \frac{\sqrt{5}}{2}$, the closest integer to $\frac{\phi^n}{\sqrt{5}}$ is equal to $\frac{\phi^n-\psi^n}{\sqrt{5}}$ which is equal to $\Fib(n)$.





\end{document}